% Based on template: templates/cv_template_en.tex

\documentclass[helvetica,english,logo,notitle,totpages,utf8]{europecv2013}
\usepackage{graphicx}
\usepackage[a4paper,top=1.2cm,left=1.2cm,right=1.2cm,bottom=2.5cm]{geometry}
\usepackage[english]{babel}
\usepackage[T1]{fontenc}
\usepackage{natbib}
\usepackage{bibentry}

\ecvpicture[width=3cm]{cv_dennis_benders_photo.jpg}

%[Tutti i campi del CV sono facoltativi. Rimuovere i campi vuoti.]
\ecvname{Dennis Benders}
\ecvaddress{7, Schoutstraat, Krimpen aan den IJssel, 2922 VT, The Netherlands}
\ecvtelephone[+31 (0)15 27 89874]{+31 (0)6 10 69 51 61}
\ecvemail{d.benders@tudelft.nl}
\ecvhomepage{\href{https://www.github.com/dbenders1}{www.github.com/dbenders1}}
\ecvlinkedin{\href{https://www.linkedin.com/in/dennisbenders}{www.linkedin.com/in/dennisbenders}}
\ecvgender{Male}
\ecvdateofbirth{01/11/1997}
\ecvnationality{Dutch}

\ecvfootnote{© European Union, 2002-2015 | http://europass.cedefop.europa.eu}
%\ecvbeforepicture{\raggedleft}
%\ecvpicture[width=2.5cm]{file-immagine-eps}
%\ecvafterpicture{\ecvspace{-37mm}}

\begin{document}
\selectlanguage{english}

\begin{europecv}
\ecvpersonalinfo[10pt]

% \ecvposition{Job applied for
% Position
% Preferred kob
% Studies applied for}{Replace with job applied for / position / preferred job / studies applied for (delete non relevant headings in left column)}

\ecvsection{Work experience}
\ecvworkexperience{2021 -- Present}{PhD Cognitive Robotics}{Delft University of Technology, Delft}{\textit{Research topic:} Safe Model Predictive Control on drones\newline\textit{Ancillary activities:} PhD representative in department, supervisor of bachelor and master students, mentor of master robotics students\newline\textit{Topics of interest:} Model Predictive Control, software engineering, sustainability}{}
\ecvworkexperience{2017 -- 2018}{Student coordinator Study Choice Check Electrical Engineering}{Delft University of Technology, Delft}{The Study Choice Check (SCC) is meant to provide high school students more in-depth information about the study. The student coordinator of the SCC has the following tasks:
\begin{itemize}
    \item taking care of the communication from university to student for the SCC days of the study Electrical Engineering
    \item guiding the students at the SCC days
    \item designing the SCC online course
\end{itemize}}{}
\ecvworkexperience{2016 -- 2017}{Student mentor}{Delft University of Technology, Delft}{A student mentor is assigned to a group of students to provide a confidential atmosphere and help them in the early phase of the study:
\begin{itemize}
    \item guiding students during their first days at the university
    \item organising weekly meetings to discuss recent topics and students' study progress
    \item being a continuously approachable person in case students encounter difficulties
\end{itemize}}{}
\ecvworkexperience{2014 -- 2015}{Helping student}{Krimpen aan den IJssel}{Working as a helping student, who is available to help other people in society. This includes functions, such as:
\begin{itemize}
    \item tutoring of high school students
    \item garden maintenance
\end{itemize}}{}
\ecvworkexperience{2011 -- 2015}{Ancillary functions}{Krimpen aan den IJssel}{Other functions performed are listed below:
\begin{itemize}
    \item summer job at a construction market (2015): refilling and scanning stock
    \item "students help students" (2012 -- 2015): tutoring of students coming from Krimpenerwaard College (once a week)
    \item korfball trainer (2011 -- 2015): training (twice a week) and coaching (once a week) a group consisting of 6-10 children at the age of 7-11 years old. This also includes taking care of the communication with parents
    \item trust pupil (2011 -- 2013): being available to students coming from Krimpenerwaard College to listen and give advice in case they are having mental problems or just want to talk about something
\end{itemize}}{}

\ecvsection{Education and training}
\ecveducation{2018 -- 2020}{Master Embedded Systems}{Delft University of Technology, Delft}{\textit{Specialisation:} Systems and Control\newline\textit{Graduation project:} Drone state estimation using brain-inspired parameter estimation theory
}{}
\ecveducation{2015 -- 2018}{Bachelor Electrical Engineering}{Delft University of Technology, Delft}{\textit{Minor:} Responsible innovation
}{}
\ecveducation{2009 -- 2015}{High School VWO graduate}{Krimpenerwaard College, Krimpen aan den IJssel}{\textit{Profile project:} "Solar scooter": adding a solar-powered electrical driveline to a normal scooter
}{}

\ecvsection{Personal skills}

\ecvmothertongue[20pt]{Dutch}
\ecvlanguageheader
\ecvlanguage{English}{C1}{C1}{C1}{C1}{C1}
\ecvlastlanguage{German}{B2}{C1}{B2}{B2}{B2}
\ecvlanguagefooter[10pt]

\ecvitem[10pt]{Communication skills}{Team work: in teams (technical project teams as well as korfball teams) I have always been the type of person who is really striving for good communication and interaction between everyone who is involved within the team and outside the team.\newline Mediating skills: as a trust pupil I was trained to lead discussions between people who were having problems with each other and to work towards a solution in a constructive way.}
\ecvitem[10pt]{Organisational / managerial skills}{When it comes to project work during my study, in most cases I am the one who takes the lead. This includes organising meetings, thinking about the software architecture, creating project plannings and working towards an atmosphere in which every team member is feeling comfortable.}
% \ecvitem[10pt]{Job-related skills}{Worked on developing the physical drone setup and corresponding software}
\ecvitem[10pt]{Computer skills}{The following skills are rated on a scale from 1 (basic level) to 3 (good level):
\begin{itemize}
    \item programming languages: C/C++ (3), Python (3), ROS (3), Matlab (2)
    \item tools: Microsoft Office (3), \LaTeX{} (3), Git (3), GanttProject (1)
    \item operating systems: Windows (3), Linux (3)
\end{itemize}}
% \ecvitem[10pt]{Other skills}{Replace with other relevant skills not already mentioned. Specify in what context they were acquired. Example:\par
% carpentry}

\ecvitem{Driving licence}{B, AM (2015)}

\ecvsection{Additional information}
\ecvitem[10pt]{Hobbies}{- sports: korfball, running and ice skating\newline- music: drums and singing}

% \ecvitem[10pt]{Publications
% Presentations
% Projects
% Conferences
% Seminars
% Honours and awards
% Memberships
% References}{Replace with relevant publications, presentations, projects, conferences, seminars, honours and awards, memberships, references. Remove headings not relevant in the left column.}

% \ecvsection{Annexes}

% \ecvitem[10pt]{}{Replace with list of documents annexed to your CV. Examples:
% \begin{itemize}
% \item copies of degrees and qualifications; 
% \item testimonial of employment or work placement;
% \item publications or research.
% \end{itemize}}

\bibliographystyle{plain}
\nobibliography{cv_dennis_benders_publications.bib} % bib file name

\ecvsection{Publications}

\ecvitem{Pub1}{\bibentry{bas2022free}}
\ecvitem{Pub2}{\bibentry{tsolakis2022colregs}}

\end{europecv}
\end{document} 
