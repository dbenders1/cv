% Based on template: templates/cv_template_en.tex

\documentclass[helvetica,dutch,logo,notitle,totpages,utf8]{europecv2013}
\usepackage{graphicx}
\usepackage[a4paper,top=1.2cm,left=1.2cm,right=1.2cm,bottom=2.5cm]{geometry}
\usepackage[english]{babel}
\usepackage[T1]{fontenc}
\usepackage{natbib}
\usepackage{bibentry}

\ecvpicture[width=112pt]{cv_dennis_benders_photo.jpg}

%[Tutti i campi del CV sono facoltativi. Rimuovere i campi vuoti.]
\ecvname{Dennis Benders}
\ecvaddress{Krimpen aan den IJssel, Nederland}
\ecvtelephone[]{+31 (0)6 10 69 51 61}
\ecvemail{d.benders@tudelft.nl}
\ecvhomepage{\href{https://www.github.com/dbenders1}{www.github.com/dbenders1}}
\ecvlinkedin{\href{https://www.linkedin.com/in/dennisbenders}{www.linkedin.com/in/dennisbenders}}
\ecvgender{Man}
\ecvdateofbirth{01-11-1997}
\ecvnationality{Nederlands}

\ecvfootnote{© Europese Unie, 2002-2015 | http://europass.cedefop.europa.eu}
%\ecvbeforepicture{\raggedleft}
%\ecvpicture[width=2.5cm]{file-immagine-eps}
%\ecvafterpicture{\ecvspace{-37mm}}

\begin{document}
\selectlanguage{english}

\begin{europecv}
\ecvpersonalinfo[10pt]

% \ecvposition{Job applied for
% Position
% Preferred kob
% Studies applied for}{Replace with job applied for / position / preferred job / studies applied for (delete non relevant headings in left column)}

\ecvsection{Werkervaring}
\ecvworkexperience{2021 -- Present}{PhD Cognitive Robotics}{Technische Universiteit Delft, Delft}{\textit{Onderzoeksonderwerp:} Veilige Model Predictive Control (MPC) algoritmes voor mobiele robots\newline\textit{Nevenactiviteiten:}
\begin{itemize}
    \item Betrokken bij het ontwerpen en bouwen van het mobiele robotica lab in de CoR afdeling (motion capture cameras, video cameras, netwerk setup, werkplekken, veiligheidsoverwegingen, etc.): zie ook \href{https://delta.tudelft.nl/article/nieuw-lab-ontwikkelt-robots-die-mensen-begrijpen}{deze} en \href{https://www.procarebv.nl/tu-delft-gebruikt-vicon-voor-tracking-robots-en-drones/}{deze} pagina's
    \item Verantwoordelijk voor het bouwen van een van de quadrotor platformen in de onderzoeksgroep
    \item Eerste PhD vertegenwoordiger in de CoR afdeling
    \item Begeleideer van 3 bachelor groepjes en 3 master studenten
    \item Menter van een groep master robotica studenten
\end{itemize}
\textit{Onderwerpen van interesse:} robotica, MPC, software engineering, duurzaamheid}{}
\ecvworkexperience{2017 -- 2018}{Student coördinator studiekeuzecheck Elektrotechniek}{Technische Universiteit Delft, Delft}{De studiekeuzecheck is ervoor bedoeld om middelbare scholieren meer gedetailleerde informatie te geven over de studie. De student coördinator van de studiekeuzechck is verantwoordelijk voor de communicatie van universiteit naar scholier, het begeleiden van de scholieren op de studiekeuzecheckdag en het ontwerpen van de bijbehorende online cursus.}{}
\ecvworkexperience{2016 -- 2017}{Student mentor}{Technische Universiteit Delft, Delft}{Een student mentor wordt toegewezen aan een groep studenten voor het creëeren van een veilige omgeving in de beginfase van de studie. Dit houdt in: het begeleiden van studenten tijdens hun eerste dagen aan de universiteit, het organiseren van wekelijkse bijeenkomsten voor het bespreken van recente onderwerpen, de voortgang van de studenten en het eerste aanspreekpunt zijn in het geval een student ergens mee zit.}{}

\ecvsection{Educatie}
\ecveducation{2018 -- 2020}{Master Embedded Systems}{Technische Universiteit Delft, Delft}{\textit{Specialisatie:} Systems and Control\newline\textit{Afstudeerproject:} Quadrotor state estimation gebruikmakende een op de hersenen geïnspireerde parameterschattingstheorie
}{}
\ecveducation{2015 -- 2018}{Bachelor Elektrotechniek}{Technische Universiteit Delft, Delft}{\textit{Minor:} Responsible innovation
}{}
\ecveducation{2009 -- 2015}{Middelbare school VWO}{Krimpenerwaard College, Krimpen aan den IJssel}{\textit{Profielwerkstuk:} "Zonnestep": het toevoegen van een op zonnepaneel gebaseerde elektrische aandrijflijn aan een step
}{}

\ecvsection{Persoonlijke vaardigheden}

\ecvmothertongue[20pt]{Nederlands}
\ecvlanguageheader
\ecvlanguage{Engels}{C1}{C1}{C1}{C1}{C1}
\ecvlastlanguage{Duits}{B2}{C1}{B2}{B2}{B2}
\ecvlanguagefooter[10pt]

\ecvitem[10pt]{Communicatieve vaardigheden}{Teamwerk: in teams (zowel technische projectteams als korfbalteams) ben ik altijd het type persoon geweest die streeft naar goede communicatie en interactie binnen het team en naar buiten toe.\newline Bemiddelingsvaardigheden: als vertrouwenspersoon op de middelbare ben ik getraind in het leiden van discussies en het toewerken naar oplossingen op een constructieve manier.}
\ecvitem[10pt]{Organisatie- en managementvaardigheden}{Als het aankomt op projectwerk, zowel tijdens de studie als de PhD, was ik altijd de peroon die snel de leiding neemt. Ik denk automatisch na over het organiseren van meetings, de software architectuur, het maken van een project planning en het toewerken naar een veilige omgeving waarin elk lid van het team zich thuis en gewaardeerd voelt.}
% \ecvitem[10pt]{Job-related skills}{Worked on developing the physical drone setup and corresponding software}
\ecvitem[10pt]{Computervaardigheden}{De volgende vaardigheden zijn beoordeeld op een schaal van 1 (basis) tot 3 (goed):
\begin{itemize}
    \item programmeertalen: C/C++ (3), Python (3), ROS (3), Matlab (3)
    \item programma's: Microsoft Office (3), \LaTeX{} (3), Git (3), GanttProject (1)
    \item besturingssystemen: Windows (3), Linux (3)
\end{itemize}}
% \ecvitem[10pt]{Other skills}{Replace with other relevant skills not already mentioned. Specify in what context they were acquired. Example:\par
% carpentry}

\ecvitem{Rijbewijs}{B, AM (2015)}

\ecvsection{Extra informatie}
\ecvitem[10pt]{Hobby's}{- sport: korfbal, hardlopen en schaatsen\newline- muziek: percussie}

% \ecvitem[10pt]{Publications
% Presentations
% Projects
% Conferences
% Seminars
% Honours and awards
% Memberships
% References}{Replace with relevant publications, presentations, projects, conferences, seminars, honours and awards, memberships, references. Remove headings not relevant in the left column.}

% \ecvsection{Annexes}

% \ecvitem[10pt]{}{Replace with list of documents annexed to your CV. Examples:
% \begin{itemize}
% \item copies of degrees and qualifications; 
% \item testimonial of employment or work placement;
% \item publications or research.
% \end{itemize}}

\bibliographystyle{plain}
\nobibliography{cv_dennis_benders_publications_tno.bib} % bib file name

\ecvsection{Publicaties}

\ecvitem{Pub1}{\bibentry{benders2024embedded}}
\ecvitem{Pub2}{\bibentry{tsolakis2022colregs}}
\ecvitem{Pub3}{\bibentry{bas2022free}}

\end{europecv}
\end{document} 
